\section{Abstract}
\label{se:abstract}
IMO has in the second generation intact stability criteria addressed the importance of ships having sufficient roll damping to avoid \emph{Parametric roll}, and large roll motions in \emph{Dead Ship condition} as well as \emph{Excessive acceleration}. Semi-empirical methods such as \emph{Ikeda's method} is widely used to predict roll damping for practical purposes, especially in the early design stage of ships, where \emph{CFD} or model tests are not feasible options. The applicability and accuracy of existing semi-empirical methods for modern hullforms are somewhat uncertain which is very unfortunate, especially for analysis of parametric roll. Just a small error in the roll damping prediction can make the difference between having no parametric roll and disaster.

The purpose with this paper is to develop a new semi-empirical method to predict roll damping for modern ships. The method is developed with machine learning techniques applied on historical roll damping data. The data is obtained from roll decay model test conducted at \emph{SSPA} during the past 15 years. The method is meant to be used by the industry for practical purposes, where roll damping is used as input to various simulation tools. The method accuracy can be estimated using cross validation, which is indeed very valuable information when the method is used in engineering applications. Improved roll damping predictions enables safer and/or more energy efficient ships.
The new method is better than similar older methods since it is based on a large set of relatively new empirical data and thereby more applicable to modern ships.
The research in this paper should however also be relevant for the research community, where the possibilities of using machine learning on a large data base of high quality hydrodynamic data is explored for one specific application.

\newpage