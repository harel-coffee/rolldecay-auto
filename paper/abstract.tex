\section*{Abstract}
\label{se:abstract}
IMO, the International Maritime Organization, has in the second generation intact stability criteria \cite{imo_finalization_2016} addressed the importance of ships having sufficient roll damping to avoid parametric roll and large roll motions in dead ship condition as well as excessive acceleration. Semi-empirical methods such as Ikeda’s method is widely used to predict roll damping for practical purposes, especially in the early design stage of ships where computational fluid dynamics or experimental model tests are not feasible options. Recent work has shown that the applicability and accuracy of Ikeda’s method for modern hull forms are somewhat uncertain which is very unfortunate, especially for analysis of parametric roll. A small error in the roll damping prediction can make the difference between having no parametric roll and disaster \cite{soder_ikeda_2019}.

The purpose of this work is to develop a new semi-empirical method to predict roll damping for modern ships. The method is developed with machine learning techniques applied on historical roll damping data. The data is obtained from roll decay model test conducted in the Maritime Dynamics Laboratory at SSPA Sweden AB (www.sspa.se) during the past 15 years. The method is meant to be used by the industry for practical purposes, where roll damping is used as input to various simulation tools. The accuracy can be estimated using cross validation, which is indeed very valuable information when the method is used in engineering applications. The method presented in this paper should, however, also be relevant for the research community where the possibilities of using machine learning on a large data base of high quality hydrodynamic data is explored. Improved roll damping predictions enables optimization for safer and more energy efficient ships. The new method is based on a large set of relatively new empirical data and should be more applicable to modern ships than older methods such as Ikeda’s method.

%\vspace*{-0.10in}
\begin{figure}[h]
  \centering
  \includegraphics{figures/rolldecay.pdf}
\end{figure}
\vspace*{-0.3in}
